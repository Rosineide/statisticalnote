% Options for packages loaded elsewhere
\PassOptionsToPackage{unicode}{hyperref}
\PassOptionsToPackage{hyphens}{url}
%
\documentclass[
]{article}
\usepackage{lmodern}
\usepackage{amssymb,amsmath}
\usepackage{ifxetex,ifluatex}
\ifnum 0\ifxetex 1\fi\ifluatex 1\fi=0 % if pdftex
  \usepackage[T1]{fontenc}
  \usepackage[utf8]{inputenc}
  \usepackage{textcomp} % provide euro and other symbols
\else % if luatex or xetex
  \usepackage{unicode-math}
  \defaultfontfeatures{Scale=MatchLowercase}
  \defaultfontfeatures[\rmfamily]{Ligatures=TeX,Scale=1}
\fi
% Use upquote if available, for straight quotes in verbatim environments
\IfFileExists{upquote.sty}{\usepackage{upquote}}{}
\IfFileExists{microtype.sty}{% use microtype if available
  \usepackage[]{microtype}
  \UseMicrotypeSet[protrusion]{basicmath} % disable protrusion for tt fonts
}{}
\makeatletter
\@ifundefined{KOMAClassName}{% if non-KOMA class
  \IfFileExists{parskip.sty}{%
    \usepackage{parskip}
  }{% else
    \setlength{\parindent}{0pt}
    \setlength{\parskip}{6pt plus 2pt minus 1pt}}
}{% if KOMA class
  \KOMAoptions{parskip=half}}
\makeatother
\usepackage{xcolor}
\IfFileExists{xurl.sty}{\usepackage{xurl}}{} % add URL line breaks if available
\IfFileExists{bookmark.sty}{\usepackage{bookmark}}{\usepackage{hyperref}}
\hypersetup{
  pdftitle={Introdução},
  hidelinks,
  pdfcreator={LaTeX via pandoc}}
\urlstyle{same} % disable monospaced font for URLs
\usepackage[margin=1in]{geometry}
\usepackage{graphicx,grffile}
\makeatletter
\def\maxwidth{\ifdim\Gin@nat@width>\linewidth\linewidth\else\Gin@nat@width\fi}
\def\maxheight{\ifdim\Gin@nat@height>\textheight\textheight\else\Gin@nat@height\fi}
\makeatother
% Scale images if necessary, so that they will not overflow the page
% margins by default, and it is still possible to overwrite the defaults
% using explicit options in \includegraphics[width, height, ...]{}
\setkeys{Gin}{width=\maxwidth,height=\maxheight,keepaspectratio}
% Set default figure placement to htbp
\makeatletter
\def\fps@figure{htbp}
\makeatother
\setlength{\emergencystretch}{3em} % prevent overfull lines
\providecommand{\tightlist}{%
  \setlength{\itemsep}{0pt}\setlength{\parskip}{0pt}}
\setcounter{secnumdepth}{-\maxdimen} % remove section numbering
\usepackage{amsmath}
\usepackage{amssymb}
\usepackage{amsthm}
\usepackage{bm}
\usepackage{bbm}
\usepackage{amsfonts}
\usepackage{mathtools} - \usepackage{tikz} - \usetikzlibrary{fit,shapes.geometric, arrows} - \usetikzlibrary{trees} - \usepackage{caption} - \usepackage{hyperref} - \usepackage[utf8]{inputenc} - \usepackage[portuguese]{babel} - \usepackage[T1]{fontenc}

\title{Introdução}
\author{}
\date{\vspace{-2.5em}}

\begin{document}
\maketitle

\newtheorem{exemplo}{Exemplo}
\newtheorem{assumption}{Assumption}
\renewcommand{\thesection}{\Alph{Seções}}
A estatística consiste numa metodologia científica para obtenção,
organização, redução, apresentação, análise e interpretação de dados
oriundos das mais variadas áreas das ciências experimentais, cujo
objetivo principal é auxiliar a tomada de decisão em situações de
incerteza, veja por exemplo Morettin
(\protect\hyperlink{ref-bussab}{2017}) e Barbetta, Reis, and Bornia
(\protect\hyperlink{ref-barbetta2004}{2004}). Informalmente, podemos
definir a ciência estatística como um conjunto de técnicas utilizadas
para estudar a condição de uma população usando informações obtidas a
partir de dados observados.

\hypertarget{de-onde-vuxeam-os-dados}{%
\paragraph{De onde vêm os dados?}\label{de-onde-vuxeam-os-dados}}

Dados são resultados de observações de algum fenômeno, podendo ser
obtidos, para análise, a partir de observações espontâneas ou por meio
de realização de experimentos planejados.

\begin{itemize}
\tightlist
\item
  Dados oriundos de observações de fenômenos quaisquer:

  \begin{itemize}
  \tightlist
  \item
    observar o desempenho natural de um novo equipamento.
  \end{itemize}
\end{itemize}

Nesse caso, o desempenho natural do equipamento é a característica que
se deseja estudar.

\begin{itemize}
\item
  Dados oriundos de experimentos planejados:
\item
  observar o desempenho de um novo equipamento, alterando de modo
  proposital alguma característica.

  Nesse caso, o interesse é estudar o desempenho do equipamento levando
  em consideração a variação da característica alterada.
\end{itemize}

A estatística, muitas vezes, é de grande utilidade quando o método
científico é utilizado para testar teoria ou hipóteses em muitas áreas
do conhecimento. Esse método pode ser resumido nos seguintes passos.

\begin{itemize}
\item
  Um problema é formulado em que, muitas vezes, uma hipótese precisa ser
  testada.

  \begin{itemize}
  \tightlist
  \item
    Para solucionar o problema, deve-se coletar informações que sejam
    relevantes, para isso pode-se formular um experimento. Em muitas
    áreas do conhecimento o planejamento do experimento não é simples,
    ou até mesmo não é possível, e uma estratégia pode ser a observação
    de algum fenômeno de interesse.
  \end{itemize}
\item
  Os resultados do experimento podem ser utilizados para se obter
  conclusões, definitivas ou não.
\item
  Os passo 2 e 3 podem ser repetidos quantas vezes forem necessárias.
\end{itemize}

\begin{verbatim}
Ao observar um equipamento que deveria estar operando, nota-se que este está parado.

- Hipotese: falta de energia elétrica.

- Faz-se a observação para verificar a hipótese

- Se não é falta de energia, outras observações e testes serão requeridos.

- Os passos serão executados até que se tenha uma conclusão, que pode ser definitiva ou não.
        
\end{verbatim}

Muitas vezes, na aplicação do método científico a estatística é uma
ferramenta indispensável, podendo ser requerida em todas as etapas.

\begin{verbatim}
    Ao observar um equipamento em operação, desconfia-se que este não está operando como deveria.

* Hipótese: o equipamento está desregulado, neste caso pode-se optar por realizar um processo de amostragem de produtos fabricados pela máquina, então faz-se necessário o planejamento de um experimento para coleta das amostras (área da estatística).

* Após o planejamento, o experimento é realizado e ferramentas da estatística podem ser aplicadas para obter alguma conclusão.

* Caso os resultados sejam inconclusivos, novos experimentos poderão ser realizados.

* Os passos podem ser executados até que se tenha uma conclusão.
\end{verbatim}

\hypertarget{planejamento-do-estudo}{%
\subsubsection{Planejamento do Estudo}\label{planejamento-do-estudo}}

É nessa esta que deve ser definida a \{\bf população\} de interesse, ou
seja deve ser feita a \textbf{Formulação do Problema}.

Em uma análise estatística, a população pode ser pensada como o conjunto
que contém todos os indivíduos, fenômenos ou resultados que se pretende
investigar,sendo bem delimitado por pelo menos uma característica
compartilhada por todos os seus elementos.

\hypertarget{refs}{}
\leavevmode\hypertarget{ref-barbetta2004}{}%
Barbetta, Pedro Alberto, Marcelo Menezes Reis, and Antonio Cezar Bornia.
2004. \emph{Estatı́stica: Para Cursos de Engenharia E Informática}. Vol.
3. Atlas São Paulo.

\leavevmode\hypertarget{ref-bussab}{}%
Morettin, WILTON OLIVEIRA, Pedro Alberto e BUSSAB. 2017.
\emph{Estatı́stica Básica}. Editora Saraiva.

\end{document}
